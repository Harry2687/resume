%-------------------------------------------------------------------------------
%	SECTION TITLE
%-------------------------------------------------------------------------------
\cvsection{Projects}


%-------------------------------------------------------------------------------
%	CONTENT
%-------------------------------------------------------------------------------
\begin{cventries}

%---------------------------------------------------------
  \cventry
    {Python}
    {\href{https://github.com/Harry2687/Messenger-Analysis}{Analysis of my Facebook Messenger chat history using latent Dirichlet allocation to identify topics within conversations \ExternalLink}}
    {Personal Project}
    {2024}
    {
      \begin{cvitems}
        \item {Used Python to extract Facebook Messenger chat data stored as JSON files.}
        \item {Implemented latent Dirichlet allocation (LDA) using the gensim module.}
        \item {Tuned the 3 hyperparameters of LDA using coherence score as a measure of model performance.}
      \end{cvitems}
    }

  \cventry
    {R (Programming Language)} % Affiliation/role
    {\href{https://harryz.netlify.app/projects/spotify-analysis/}{Analysis of my Spotify streaming history using k-means clustering to categorise tracks \ExternalLink}} % Organization/group
    {Personal Project} % Location
    {2024} % Date(s)
    {
      \begin{cvitems} % Description(s) of experience/contributions/knowledge
        \item {Used R to extract Spotify streaming history along with Spotify's developer API to query track features.}
        \item {Used Spotify track features to apply k-means clustering algorithm to tracks in streaming history.}
        \item {Implemented brute force method of feature selection and parallel processing to optimise computation time.}
        \item {Visualised clustering results and streaming trends using Shiny applications.}
      \end{cvitems}
    }

  \cventry
    {R (Programming Language)} % Affiliation/role
    {\href{https://www.researchgate.net/publication/372915363_Comparing_Stochastic_and_Constant_Volatility_Returns_Distributions_using_the_Heston_Model}{Comparing stochastic and constant volatility returns distributions using the Heston model (Actuarial Science Honours Dissertation) \ExternalLink}} % Organization/group
    {Curtin University} % Location
    {2023} % Date(s)
    {
      \begin{cvitems} % Description(s) of experience/contributions/knowledge
        \item {Used stochastic modelling to compare constant and stochastic volatility under geometric Brownian motion based on independent research.}
        \item {Used R to optimise parameterisation and simulation via parallel processing.}
        \item {Completed written research report and presented seminar presentation to supervisors.}
      \end{cvitems}
    }

  \cventry
    {R (Programming Language)} % Affiliation/role
    {Validation of additional factors in the Capital Asset Pricing Model (CAPM) using machine learning algorithms} % Organization/group
    {Curtin University} % Location
    {2022} % Date(s)
    {
      \begin{cvitems} % Description(s) of experience/contributions/knowledge
        \item {Used Elastic Net and Random Forest models combined with hyperparameter tuning to determine whether additional covariates in CAPM improved empirical predictive performance.}
        \item {Collaborated with team members to write code, compile written report, and present results.}
      \end{cvitems}
    }
%---------------------------------------------------------
\end{cventries}
