%!TEX TS-program = xelatex
%!TEX encoding = UTF-8 Unicode
% Awesome CV LaTeX Template for CV/Resume
%
% This template has been downloaded from:
% https://github.com/posquit0/Awesome-CV
%
% Author:
% Claud D. Park <posquit0.bj@gmail.com>
% http://www.posquit0.com
%
% Template license:
% CC BY-SA 4.0 (https://creativecommons.org/licenses/by-sa/4.0/)
%


%-------------------------------------------------------------------------------
% CONFIGURATIONS
%-------------------------------------------------------------------------------
% A4 paper size by default, use 'letterpaper' for US letter
\documentclass[11pt, a4paper]{awesome-cv}

% Configure page margins with geometry
\geometry{left=1.4cm, top=.8cm, right=1.4cm, bottom=1.8cm, footskip=.5cm}

% Color for highlights
% Awesome Colors: awesome-emerald, awesome-skyblue, awesome-red, awesome-pink, awesome-orange
%                 awesome-nephritis, awesome-concrete, awesome-darknight
\colorlet{awesome}{awesome-red}
% Uncomment if you would like to specify your own color
% \definecolor{awesome}{HTML}{3E6D9C}

% Colors for text
% Uncomment if you would like to specify your own color
% \definecolor{darktext}{HTML}{414141}
% \definecolor{text}{HTML}{333333}
% \definecolor{graytext}{HTML}{5D5D5D}
% \definecolor{lighttext}{HTML}{999999}
% \definecolor{sectiondivider}{HTML}{5D5D5D}

% Set false if you don't want to highlight section with awesome color
\setbool{acvSectionColorHighlight}{false}

\usepackage{tikz}

\newcommand{\ExternalLink}{
    \tikz[x=1.2ex, y=1.2ex, baseline=-0.05ex]{% 
        \begin{scope}[x=1ex, y=1ex]
            \clip (-0.1,-0.1) 
                --++ (-0, 1.2) 
                --++ (0.6, 0) 
                --++ (0, -0.6) 
                --++ (0.6, 0) 
                --++ (0, -1);
            \path[draw, 
                line width = 0.5, 
                rounded corners=0.5] 
                (0,0) rectangle (1,1);
        \end{scope}
        \path[draw, line width = 0.5] (0.5, 0.5) 
            -- (1, 1);
        \path[draw, line width = 0.5] (0.6, 1) 
            -- (1, 1) -- (1, 0.6);
        }
    }

% If you would like to change the social information separator from a pipe (|) to something else
\renewcommand{\acvHeaderSocialSep}{\quad\textbar\quad}


%-------------------------------------------------------------------------------
%	PERSONAL INFORMATION
%	Comment any of the lines below if they are not required
%-------------------------------------------------------------------------------
% Available options: circle|rectangle,edge/noedge,left/right
% \photo[rectangle,edge,right]{./examples/profile}
\name{Harry}{Zhong}
\position{Actuary (AIAA){\enskip\cdotp\enskip}Data Analyst}
\address{Perth WA}

\mobile{0403 041 075}
\email{h.zhong2687@gmail.com}
%\dateofbirth{January 1st, 1970}
\homepage{harryz.netlify.app}
\github{harry2687}
\linkedin{harry2687}
% \gitlab{gitlab-id}
% \stackoverflow{SO-id}{SO-name}
% \twitter{@twit}
% \skype{skype-id}
% \reddit{reddit-id}
% \medium{madium-id}
% \kaggle{kaggle-id}
% \hackerrank{hackerrank-id}
% \googlescholar{googlescholar-id}{name-to-display}
%% \firstname and \lastname will be used
% \googlescholar{googlescholar-id}{}
% \extrainfo{extra information}

% \quote{``Be the change that you want to see in the world."}


%-------------------------------------------------------------------------------
\begin{document}

% Print the header with above personal information
% Give optional argument to change alignment(C: center, L: left, R: right)
\makecvheader[C]

% Print the footer with 3 arguments(<left>, <center>, <right>)
% Leave any of these blank if they are not needed
% \makecvfooter
%   {\today}
%   {Byungjin Park~~~·~~~Résumé}
%   {\thepage}


%-------------------------------------------------------------------------------
%	CV/RESUME CONTENT
%	Each section is imported separately, open each file in turn to modify content
%-------------------------------------------------------------------------------
% %-------------------------------------------------------------------------------
%	SECTION TITLE
%-------------------------------------------------------------------------------
\cvsection{Summary}


%-------------------------------------------------------------------------------
%	CONTENT
%-------------------------------------------------------------------------------
\begin{cvparagraph}

%---------------------------------------------------------
DevOps Engineer at fintech \& blockchain company Dunamu which is known for operating Upbit, the largest cryptocurrency exchange in Korea. Have led growth at infrastructure departments in two fintech companies as lead engineer and founding member. 12+ years of diverse software engineering experience with specialties in software architecture design, infrastructure operation, backend development, and security engineering.

Love to contribute to open sources and tech communities by sharing knowledge and experience. Prefers a command line interface environment as a big fan of Vim, Linux, and macOS. Always trying to customize to find the most optimal environment. Interested in devising a better problem-solving method for challenging tasks, and learning new technologies and tools.
\end{cvparagraph}

%-------------------------------------------------------------------------------
%	SECTION TITLE
%-------------------------------------------------------------------------------
\cvsection{Work Experience}


%-------------------------------------------------------------------------------
%	CONTENT
%-------------------------------------------------------------------------------
\begin{cventries}

%---------------------------------------------------------
  \cventry
    {Data Intelligence Analyst} % Job title
    {EBM Insurance \& Risk} % Organization
    {Perth} % Location
    {Nov 2022 - Present} % Date(s)
    {
      \begin{cvitems} % Description(s) of tasks/responsibilities
        \item {Gained a strong understanding of data analysis principles (relational databases, query languages, modelling) and applied them to a broad range of financial and insurance applications (financial reporting, scenario analysis, cashflow projection).}
        \item {Key contributor to the internal development of data analysis applications using \textbf{SQL} and \textbf{R}.}
        \item {Introduced the use of \textbf{Shiny} applications for data visualisation and report distribution, and \textbf{Git} for version control in \textbf{R}.}
        \item {Modelled annual income of client portfolios in \textbf{Microsoft Excel} for discounted cash flow valuations.}
        \item {Delivered monthly income reports using \textbf{VBA} to automate repetitive tasks.}
      \end{cvitems}
    }

%---------------------------------------------------------
\end{cventries}

%-------------------------------------------------------------------------------
%	SECTION TITLE
%-------------------------------------------------------------------------------
\cvsection{Education}


%-------------------------------------------------------------------------------
%	CONTENT
%-------------------------------------------------------------------------------
\begin{cventries}

%---------------------------------------------------------
  \cventry
    {Actuary Program} % Degree
    {Actuaries Institute} % Institution
    {Perth (Online)} % Location
    {Jul 2023 - Oct 2023} % Date(s)
    {
      \begin{cvitems} % Description(s) bullet points
        \item {Obtained \textbf{Associate Actuary (AIAA)} designation in December 2023.}
        \item {Passed \textbf{Asset Liability Management} and \textbf{Communication, Modelling and Professionalism}.}
      \end{cvitems}
    }

    \cventry
    {Bachelor of Science (Actuarial Science) (Honours)} % Degree
    {Curtin University} % Institution
    {Bentley Campus} % Location
    {Feb 2019 - Jun 2023} % Date(s)
    {
      \begin{cvitems} % Description(s) bullet points
        \item {84\% Course Weighted Average.}
        \item {82\% Dissertation Grade.}
        \item {Completed \textbf{Data Analytics Principles} and \textbf{Actuarial Control Cycle} subjects as part of the Actuaries Institute Actuary Program.}
        \item {Obtained all Actuaries Institute \textbf{Foundation Program} exemptions.}
        \item {Recipient of the Curtin Excellence Scholarship.}
      \end{cvitems}
    }

%---------------------------------------------------------
\end{cventries}

%-------------------------------------------------------------------------------
%	SECTION TITLE
%-------------------------------------------------------------------------------
\cvsection{Skills}

%-------------------------------------------------------------------------------
%	CONTENT
%-------------------------------------------------------------------------------
\begin{cvskills}

%---------------------------------------------------------
  \cvskill
    {Advanced} % Category
    {R, SQL, Visual Basic for Applications (VBA), Microsoft Excel.} % Skills

%---------------------------------------------------------
  \cvskill
    {Intermediate} % Category
    {Python, Power BI, Microsoft Office suite (Word, PowerPoint, Outlook), \LaTeX.} % Skills

%---------------------------------------------------------
\end{cvskills}
%-------------------------------------------------------------------------------
%	SECTION TITLE
%-------------------------------------------------------------------------------
\cvsection{Projects}


%-------------------------------------------------------------------------------
%	CONTENT
%-------------------------------------------------------------------------------
\begin{cventries}

%---------------------------------------------------------
  \cventry
    {Python}
    {\href{https://github.com/Harry2687/Gender-Prediction}{Training a convolutional neural network to classify facial features using PyTorch \ExternalLink}}
    {Personal Project}
    {2024}
    {
      \begin{cvitems}
        \item {Used PyTorch to define and implement custom convolutional neural network used to classify facial features using the CelebA dataset.}
        \item {Tuned hyperparameters (batch size, kernel sizes, and layer composition) to optimise convergence speed and loss measure.}
        \item {Achieved 95\% accuracy in training and testing splits of dataset.}
      \end{cvitems}
    }

  \cventry
    {Python}
    {\href{https://github.com/Harry2687/Messenger-Analysis}{Analysis of my Facebook Messenger chat history using latent Dirichlet allocation to identify topics within conversations \ExternalLink}}
    {Personal Project}
    {2024}
    {
      \begin{cvitems}
        \item {Used Python to extract Facebook Messenger chat data stored as JSON files.}
        \item {Implemented latent Dirichlet allocation (LDA) using the gensim module.}
        \item {Tuned the 3 hyperparameters of LDA using coherence score as a measure of model performance.}
      \end{cvitems}
    }

  \cventry
    {R (Programming Language)} % Affiliation/role
    {\href{https://harryz.netlify.app/projects/spotify-analysis/}{Analysis of my Spotify streaming history using k-means clustering to categorise tracks \ExternalLink}} % Organization/group
    {Personal Project} % Location
    {2024} % Date(s)
    {
      \begin{cvitems} % Description(s) of experience/contributions/knowledge
        \item {Used R to extract Spotify streaming history along with Spotify's developer API to query track features.}
        \item {Used Spotify track features to apply k-means clustering algorithm to tracks in streaming history.}
        \item {Implemented brute force method of feature selection and parallel processing to optimise computation time.}
        \item {Visualised clustering results and streaming trends using Shiny applications.}
      \end{cvitems}
    }

  \cventry
    {R (Programming Language)} % Affiliation/role
    {\href{https://www.researchgate.net/publication/372915363_Comparing_Stochastic_and_Constant_Volatility_Returns_Distributions_using_the_Heston_Model}{Comparing stochastic and constant volatility returns distributions using the Heston model (Actuarial Science Honours Dissertation) \ExternalLink}} % Organization/group
    {Curtin University} % Location
    {2023} % Date(s)
    {
      \begin{cvitems} % Description(s) of experience/contributions/knowledge
        \item {Used stochastic modelling to compare constant and stochastic volatility under geometric Brownian motion based on independent research.}
        \item {Used R to optimise parameterisation and simulation via parallel processing.}
        \item {Completed written research report and presented seminar presentation to supervisors.}
      \end{cvitems}
    }
%---------------------------------------------------------
\end{cventries}


%-------------------------------------------------------------------------------
\end{document}
